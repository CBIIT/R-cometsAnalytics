\documentclass[]{book}
\usepackage{lmodern}
\usepackage{amssymb,amsmath}
\usepackage{ifxetex,ifluatex}
\usepackage{fixltx2e} % provides \textsubscript
\ifnum 0\ifxetex 1\fi\ifluatex 1\fi=0 % if pdftex
  \usepackage[T1]{fontenc}
  \usepackage[utf8]{inputenc}
\else % if luatex or xelatex
  \ifxetex
    \usepackage{mathspec}
  \else
    \usepackage{fontspec}
  \fi
  \defaultfontfeatures{Ligatures=TeX,Scale=MatchLowercase}
\fi
% use upquote if available, for straight quotes in verbatim environments
\IfFileExists{upquote.sty}{\usepackage{upquote}}{}
% use microtype if available
\IfFileExists{microtype.sty}{%
\usepackage{microtype}
\UseMicrotypeSet[protrusion]{basicmath} % disable protrusion for tt fonts
}{}
\usepackage[margin=1in]{geometry}
\usepackage{hyperref}
\hypersetup{unicode=true,
            pdftitle={COMETS Analytics Tutorial},
            pdfauthor={CAT Work group},
            pdfborder={0 0 0},
            breaklinks=true}
\urlstyle{same}  % don't use monospace font for urls
\usepackage{longtable,booktabs}
\usepackage{graphicx,grffile}
\makeatletter
\def\maxwidth{\ifdim\Gin@nat@width>\linewidth\linewidth\else\Gin@nat@width\fi}
\def\maxheight{\ifdim\Gin@nat@height>\textheight\textheight\else\Gin@nat@height\fi}
\makeatother
% Scale images if necessary, so that they will not overflow the page
% margins by default, and it is still possible to overwrite the defaults
% using explicit options in \includegraphics[width, height, ...]{}
\setkeys{Gin}{width=\maxwidth,height=\maxheight,keepaspectratio}
\IfFileExists{parskip.sty}{%
\usepackage{parskip}
}{% else
\setlength{\parindent}{0pt}
\setlength{\parskip}{6pt plus 2pt minus 1pt}
}
\setlength{\emergencystretch}{3em}  % prevent overfull lines
\providecommand{\tightlist}{%
  \setlength{\itemsep}{0pt}\setlength{\parskip}{0pt}}
\setcounter{secnumdepth}{5}
% Redefines (sub)paragraphs to behave more like sections
\ifx\paragraph\undefined\else
\let\oldparagraph\paragraph
\renewcommand{\paragraph}[1]{\oldparagraph{#1}\mbox{}}
\fi
\ifx\subparagraph\undefined\else
\let\oldsubparagraph\subparagraph
\renewcommand{\subparagraph}[1]{\oldsubparagraph{#1}\mbox{}}
\fi

%%% Use protect on footnotes to avoid problems with footnotes in titles
\let\rmarkdownfootnote\footnote%
\def\footnote{\protect\rmarkdownfootnote}

%%% Change title format to be more compact
\usepackage{titling}

% Create subtitle command for use in maketitle
\newcommand{\subtitle}[1]{
  \posttitle{
    \begin{center}\large#1\end{center}
    }
}

\setlength{\droptitle}{-2em}

  \title{COMETS Analytics Tutorial}
    \pretitle{\vspace{\droptitle}\centering\huge}
  \posttitle{\par}
    \author{CAT Work group}
    \preauthor{\centering\large\emph}
  \postauthor{\par}
      \predate{\centering\large\emph}
  \postdate{\par}
    \date{2018-09-22}

\usepackage{color}
\usepackage{marginnote}

\begin{document}
\maketitle

{
\setcounter{tocdepth}{1}
\tableofcontents
}
\chapter*{Introduction}\label{introduction}
\addcontentsline{toc}{chapter}{Introduction}

\href{https://epi.grants.cancer.gov/comets/}{COMETS}, the
\textbf{CO}nsortium of \textbf{MET}abolomics \textbf{S}tudies, has a
major goal to produce jointly coordinated, multi-cohort, high-impact
publications devoted to advancing the methods and scientific
understanding of the human metabolome and its relationship to disease
etiology and prognosis. Since its inception in 2014, COMETS has grown to
include more than 50 cohorts worldwide.

Comets Analytics serves as the infrastructure to facilitate and
coordinate data analysis efforts. This tutorial gives an overview and
detailed instructions on the use of
\href{http://www.comets-analytics.org}{COMETS Analytics} in support of
the \href{https://epi.grants.cancer.gov/comets/}{COMETS Consortium}.

Prepared by the COMETS Tutorial Group:

Ella Temprosa

Steve Moore

Oana Zeleznik

Laura Trijsburg

Rachel Kelly

Errika Loftfield

Kathleen McClain

Kaitlyn Mazzili

\section*{Overview}\label{overview}
\addcontentsline{toc}{section}{Overview}

Each cycle of analyses consists of 5 steps described below to bundle
multiple projects to minimize burden on cohort analytic resources.\\

This tutorial serves as a guide through this cycle. You can register for
access to Comets Analytics using the instructions in Chapter
\ref{register} \protect\hyperlink{register}{Registration} For a quick
walk-through the cycle, you can use the sample input file and follow the
directions in Chapter \ref{quickstart} {[}Get Started with Sample
File{]}. Once you are ready to conduct your cohort-specific analyses,
see the Chapter \ref{cohort} {[}Conduct Cohort Data Analyses{]}. For
more technical details, refer to the Chapter \ref{manual}
\protect\hyperlink{manual}{Manual} and {[}FAQ{]} for anwers to commonly
asked questions.

\chapter{Get Started with the Sample File}\label{quickstart}

This chapter guides you through the features of COMETS Analytics using
the sample file to provide an introduction to the flow of analyses.

\section{Sample Input File}\label{sample-input-file}

There is a sample file located on the website that will serve as your
data input template. After logging in, navigate to the `Correlate' tab.
On the lower left side, there is text stating `Download Sample Input'.
Click this text, and the sample file will download. This template was
designed to conduct analyses of age and metabolite, and BMI and
metabolite correlations. It illustrates how your input data should be
formatted, and includes practice data for \textgreater{}600 metabolites
and key covariates such as age and gender. Using this sample file as
your input data will allow you to practice using the app functions and
familiarize yourself with its output. You may want to begin by
downloading and then inputting this sample file to the website to
familiarize yourself with how the data analysis works.

\section{Quick Analysis}\label{quick-analysis}

To upload your input file to the COMETS Analytics website, please follow
these steps:

\begin{enumerate}
\def\labelenumi{\arabic{enumi}.}
\tightlist
\item
  Select the `Correlate' tab.
\item
  Specify your cohort from the dropdown menu.
\item
  Choose your input data file, formatted as described above, using the
  `Choose File' button.
\item
  Once you have uploaded your file the `Check Integrity' button will
  activate, use this button to check the integrity of your data input
  file. The integrity check results are displayed in the right panel.
  The integrity check results contain a summary of the input data,
  including the number of measured metabolites (as defined by the user
  in the \emph{Metabolites} sheet), the number of cohort subjects and
  covariates (as defined by the user in the \emph{SubjectData} sheet),
  and the number of subjects by the number of metabolites measured (as
  defined by the user in the \emph{SubjectMetabolites} sheet).
  Additionally, a harmonization summary is displayed that includes
  number of harmonized and non-harmonized metabolites as well as number
  of metabolites with variance equal to zero or with more than 25\% of
  values represented by minimum values. Finally, histograms illustrating
  the distributions of the variance and the number of missing values are
  plotted (figures not shown here). Integrity checks are run to ensure
  the appropriate data are presented for analysis. If integrity checks
  fail, please email Kaitlyn Mazilli
  (\href{mailto:kaitlyn.mazzilli@nih.gov}{\nolinkurl{kaitlyn.mazzilli@nih.gov}}).
  If all integrity checks are passed, as indicated in a green banner
  above the summary output, please click `Download Results' in the right
  corner and email the file to
  \href{mailto:comets.analytics@gmail.com}{\nolinkurl{comets.analytics@gmail.com}}
  which is managed by Nathan Appel at IMS.
\end{enumerate}

\hypertarget{manual}{\chapter{Manual}\label{manual}}

\emph{Important} features and technical details are described in this
chapter.

\hypertarget{register}{\section{Registration}\label{register}}

{Users are required to LOG IN using an existing Facebook or Google
account, or can sign up for a login using the ``SIGN UP'' button. Your
gmail account through your university may be used for authentication. }

\marginnote{
<span class='texta'>**a. Login**</span> is initially active where you may use the 3rd party login or your account.
<span class='textb'>**b. Sign-Up** </span> click on this tab to register for an account.
<span class='textc'>**c. 3rd party login**</span> click on the facebook or google icons to login using your credentials from your account.
<span class='textd'>**d. Forget password**</span> to reset your password. You are required to provide the email associated with your account. 
<span class='texte'>**e. Proceed**</span> to the web app by clicking on the orange button.
}

\section{Data Preparation}\label{data-preparation}

Standard input using excel, a widely accessible format, is required for
COMETS Analytics and can be created from various data file formats. Data
Integrity checks for input errors provide meaningful and actionable
messages to users for analyses to proceed. See chapter \ref{input}
{[}Step 1 Data Preparation{]}

\subsection{Create Input}\label{create-input}

\section{Harmonization details}\label{details-harmonize}

The names of metabolites from each cohort has been mapped or harmonized
to a common name. This important stage facilitates comparison of
metabolites across different studies that use different platforms and/or
naming conventions.

\subsection{Preharmonization}\label{preharmonization}

Prior to conducting your cohort analyses, xxx.

\section{Integrity checks}\label{integrity-checks}

Prior to running the models for analyses, CA conducts multi-level checks
to ensure data and models are appropriate for analyses.

\subsection{Data}\label{data}

\subsection{Models}\label{models}

\section{Correlation Analyses}\label{correlation-analyses}

\subsection{Details}\label{details}

\subsection{Output}\label{output}


\end{document}
